\documentclass[12pt]{report}
\usepackage[spanish,es-nosectiondot,es-lcroman]{babel}
\usepackage{siunitx}
\usepackage[utf8]{inputenc}
\usepackage{amsmath}
\usepackage{float}
\usepackage{listings}
\usepackage{xcolor}
\usepackage{amssymb}


\usepackage{graphicx}
\usepackage{hyperref}
\usepackage{geometry}
%Configuracion para el . en decimales
\sisetup{output-decimal-marker = {.}}
% Configuración para el código
\lstset{
	language=Python,
	basicstyle=\ttfamily\footnotesize,
	numbers=left,
	numberstyle=\tiny\color{gray},
	stepnumber=1,
	numbersep=10pt,
	backgroundcolor=\color{white},
	showspaces=false,
	showstringspaces=false,
	showtabs=false,
	frame=single,
	rulecolor=\color{black},
	tabsize=4,
	captionpos=b,
	breaklines=true,
	breakatwhitespace=false,
	linewidth=\linewidth,
	keepspaces=true,
	columns=flexible,
	keywordstyle=\bfseries\color{blue},
	commentstyle=\itshape\color{lightgray},
	stringstyle=\color{red},
	escapeinside={\%*}{*)},
}

% Configuración de los márgenes
\geometry{
	left=2cm,   % Margen izquierdo
	right=2cm,  % Margen derecho
	top=2cm,    % Margen superior
	bottom=2cm  % Margen inferior
}

% Title Page
\title{
	\begin{center}
		Tarea 1\\
		Redes de computadoras y protocolos de comunicación I
		\newline
		Sondas Voyager
	\end{center}
}
\author{Salazar Martinez Miguel Angel}
\begin{document}
	\renewcommand{\arraystretch}{1.3}
	
	\maketitle

\section*{Comunicación con las sondas Voyager}

\subsection*{1. Comunicación}
\begin{itemize}
	\item \textbf{Deep Space Network (DSN):} La comunicación con las sondas Voyager se realiza a través de la Red del Espacio Profundo (DSN), una red internacional de antenas gigantes administrada por la NASA. La DSN permite el envío y recepción de señales de radio entre la Tierra y las sondas, a pesar de las enormes distancias involucradas.
	\item \textbf{Señales de Radio:} Las sondas envían datos a la Tierra utilizando señales de radio, que viajan a la velocidad de la luz. Debido a la distancia, puede tomar varias horas para que una señal viaje de la Tierra a las sondas y viceversa.
\end{itemize}

\subsection*{2. Tecnologías de la Información Involucradas}
\begin{itemize}
	\item \textbf{Transmisores de Radio de Alta Potencia:} Las sondas están equipadas con transmisores de radio que envían señales a la Tierra.
	\item \textbf{Antenas de Alto Ganancia:} Estas antenas aseguran que las señales de radio se concentren en un haz estrecho para maximizar la fuerza de la señal en largas distancias.
	\item \textbf{Receptores de Radio Sensibles:} En la Tierra, la DSN utiliza receptores muy sensibles para captar las débiles señales que llegan desde las sondas.
\end{itemize}

\subsection*{3. Energía y Consumo}
\begin{itemize}
	\item \textbf{Generadores Termoeléctricos de Radioisótopos (RTG):} Las sondas Voyager utilizan RTGs, que convierten el calor generado por el decaimiento de materiales radiactivos (como el plutonio-238) en electricidad. Esto proporciona una fuente de energía confiable y duradera, esencial para la operación continua de las sondas en el espacio profundo.
	\item \textbf{Optimización de Consumo:} Dada la limitada cantidad de energía disponible, la gestión eficiente del consumo energético es crucial. La energía se utiliza principalmente para mantener los instrumentos científicos y los sistemas de comunicación activos.
\end{itemize}

\section*{Modelo en Capas para la Comunicación de las Sondas Voyager}

\subsection*{1. Capa Física}
\begin{itemize}
	\item \textbf{Antenas de Alto Ganancia:} Las sondas utilizan antenas parabólicas de alto ganancia para enviar y recibir señales de radio. En la Tierra, la DSN usa antenas gigantes para capturar estas señales.
	\item \textbf{Frecuencias de Radio:} Se utilizan frecuencias de radio específicas para minimizar la interferencia y maximizar la penetración en el espacio profundo.
\end{itemize}

\subsection*{2. Capa de Enlace de Datos}
\begin{itemize}
	\item \textbf{Codificación y Modulación:} Los datos se codifican para corrección de errores y se modulan para transmisión a través de ondas de radio. El sistema utiliza técnicas como \textbf{PSK (Phase Shift Keying)} para la modulación.
	\item \textbf{Telemetría y Comandos:} Los datos científicos y de telemetría se empaquetan en tramas que son enviadas a la Tierra. No hay un protocolo de sesión para confirmar la recepción, sino que se envían datos de forma continua.
\end{itemize}

\subsection*{3. Capa de Red}
\begin{itemize}
	\item \textbf{Rutas Directas:} Dado que la comunicación es punto a punto entre la sonda y la DSN, no hay necesidad de enrutar los datos a través de múltiples nodos. La "red" consiste simplemente en la trayectoria de radio desde la sonda hasta la Tierra.
	\item \textbf{Dirección de Paquetes:} Cada paquete de datos tiene un identificador que asegura que sea procesado por los sistemas correctos en la Tierra.
\end{itemize}

\subsection*{4. Capa de Transporte}
\begin{itemize}
	\item \textbf{Sin Conexión Persistente:} No hay un protocolo como TCP que maneje la sesión o el estado de conexión debido al largo retraso en la comunicación (hasta 22 horas ida y vuelta). Los datos se envían de manera unidireccional, confiando en la robustez de la capa de enlace de datos para la integridad.
\end{itemize}

\subsection*{5. Capa de Aplicación}
\begin{itemize}
	\item \textbf{Datos Científicos y Telemetría:} La información enviada incluye datos científicos (imágenes, lecturas de instrumentos) y telemetría (estado de la sonda, niveles de energía, etc.).
	\item \textbf{Comandos de la Tierra:} La sonda recibe comandos para ajustar operaciones o cambiar el estado de los instrumentos, pero no verifica la recepción de cada comando debido a la imposibilidad de mantener una sesión continua.
\end{itemize}

\subsection*{Referencias para Investigar}

	\begin{itemize}
		\item \url{https://science.nasa.gov/mission/voyager}
		\item \url{https://www.nasa.gov/directorates/somd/space-communications-navigation-program/what-is-the-deep-space-network/}
		\item \url{https://www.ngenespanol.com/el-espacio/que-son-las-voyager-1-y-2-y-cuanto-tiempo-llevan-en-el-espacio-sondas/}
	\end{itemize}
	
\end{document}

