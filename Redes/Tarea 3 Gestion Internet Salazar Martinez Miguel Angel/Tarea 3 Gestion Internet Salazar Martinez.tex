\documentclass[12pt]{report}
\usepackage[spanish,es-nosectiondot,es-lcroman]{babel}
\usepackage{siunitx}
\usepackage[utf8]{inputenc}
\usepackage{amsmath}
\usepackage{float}
\usepackage{listings}
\usepackage{xcolor}
\usepackage{amssymb}
\usepackage{graphicx}
\usepackage{hyperref}
\usepackage{geometry}
\usepackage[backend=biber,style=ieee]{biblatex}
\addbibresource{./citas/citas.bib}
%Configuracion para el . en decimales
\sisetup{output-decimal-marker = {.}}
% Configuración para el código
\lstset{
	language=Python,
	basicstyle=\ttfamily\footnotesize,
	numbers=left,
	numberstyle=\tiny\color{gray},
	stepnumber=1,
	numbersep=10pt,
	backgroundcolor=\color{white},
	showspaces=false,
	showstringspaces=false,
	showtabs=false,
	frame=single,
	rulecolor=\color{black},
	tabsize=4,
	captionpos=b,
	breaklines=true,
	breakatwhitespace=false,
	linewidth=\linewidth,
	keepspaces=true,
	columns=flexible,
	keywordstyle=\bfseries\color{blue},
	commentstyle=\itshape\color{lightgray},
	stringstyle=\color{red},
	escapeinside={\%*}{*)},
}

% Configuración de los márgenes
\geometry{
	left=2cm,   % Margen izquierdo
	right=2cm,  % Margen derecho
	top=2cm,    % Margen superior
	bottom=2cm  % Margen inferior
}

% Title Page
\title{
	\begin{center}
		Tarea 3\\
		Investigación sobre estructura, organización y gestión del Internet\\
		Redes de computadoras y protocolos de comunicación I
		
	\end{center}
}
\author{Salazar Martinez Miguel Angel}
\begin{document}
	\renewcommand{\arraystretch}{1.3}
	
	\maketitle

\section*{Respuestas en el Contexto de Redes y Telecomunicaciones}

\subsection*{1. ¿Cuál es la estructura jerárquica en la que se conforma el Internet?}

La estructura jerárquica de Internet se organiza en varios niveles, comúnmente denominados \textit{Tiers}:

\begin{itemize}
	\item \textbf{Proveedores de Nivel 1 (Tier 1 ISPs)}: Son grandes proveedores internacionales capaces de alcanzar cualquier red de Internet sin necesidad de pagar por tránsito. Ejemplos incluyen AT\&T, Deutsche Telekom y Telefónica. Estos proveedores poseen y operan la columna vertebral de Internet, conectándose directamente entre sí y con numerosos proveedores de nivel inferior.
	
	\item \textbf{Proveedores de Nivel 2 (Tier 2 ISPs)}: Suelen ser proveedores regionales o nacionales que se conectan a varios proveedores de nivel 1 y otros proveedores de nivel 2. Aunque pueden tener acuerdos de intercambio de tráfico (\textit{peering}) con otros proveedores de nivel 2, generalmente pagan a los proveedores de nivel 1 por el tránsito para alcanzar ciertas partes de Internet.
	
	\item \textbf{Proveedores de Nivel 3 (Tier 3 ISPs locales)}: Son proveedores locales que ofrecen servicios de acceso a Internet directamente a usuarios finales y pequeñas empresas. Estos proveedores suelen comprar tránsito a proveedores de nivel 2 o nivel 1 para conectarse al resto de Internet.
\end{itemize}

Además de esta jerarquía de proveedores, las redes se clasifican según su alcance geográfico:

\begin{itemize}
	\item \textbf{Red de Área Local (LAN)}: Cubre un área geográfica limitada, como una oficina o un edificio. Permite la interconexión de dispositivos en distancias cortas, ofreciendo altas velocidades de transmisión y facilidad de mantenimiento.
	
	\item \textbf{Red de Área Metropolitana (MAN)}: Se extiende por una ciudad o una región metropolitana, conectando múltiples redes LAN. Es adecuada para interconectar oficinas en una ciudad, proporcionando una velocidad de transmisión moderada.
	
	\item \textbf{Red de Área Amplia (WAN)}: Cubre áreas geográficas extensas, como países o continentes. Interconecta redes LAN y MAN, utilizando tecnologías como líneas telefónicas, enlaces de microondas y conexiones por satélite. Aunque ofrece un alcance amplio, suele tener velocidades de transmisión más bajas y mayor complejidad en su diseño y mantenimiento.
\end{itemize}

\subsection*{2. ¿Qué protocolos de comunicación son utilizados dentro de cada nivel de la estructura y para la interacción de los distintos niveles?}

En la estructura jerárquica de Internet, cada nivel utiliza una variedad de protocolos de comunicación que permiten el correcto funcionamiento de la red y la interacción entre los diferentes niveles.

\begin{itemize}
	\item \textbf{Nivel de Acceso:}
	\begin{itemize}
		\item \textit{Ethernet}: Utilizado en redes LAN para la transmisión de datos en entornos cableados.
		\item \textit{Wi-Fi (802.11)}: Protocolo de red inalámbrica utilizado para conexiones locales.
		\item \textit{PPP (Point-to-Point Protocol)}: Utilizado para la transmisión de datos a través de conexiones punto a punto, como líneas telefónicas.
	\end{itemize}
	
	\item \textbf{Nivel de Distribución:}
	\begin{itemize}
		\item \textit{IP (Internet Protocol)}: Fundamental para el direccionamiento y la enrutación de paquetes de datos en redes IP.
		\item \textit{DHCP (Dynamic Host Configuration Protocol)}: Asigna direcciones IP dinámicamente a dispositivos en la red.
		\item \textit{DNS (Domain Name System)}: Traduce nombres de dominio legibles por humanos a direcciones IP.
	\end{itemize}
	
	\item \textbf{Nivel de Red Troncal:}
	\begin{itemize}
		\item \textit{BGP (Border Gateway Protocol)}: Protocolo de enrutamiento utilizado para el intercambio de información de enrutamiento entre sistemas autónomos en Internet.
		\item \textit{MPLS (Multiprotocol Label Switching)}: Utilizado para dirigir y gestionar el tráfico de red de manera eficiente, especialmente en grandes redes troncales.
	\end{itemize}
\end{itemize}

\noindent \textbf{Interacción entre Niveles:}
\begin{itemize}
	\item \textit{IP (Internet Protocol)}: Utilizado como base para la comunicación entre todos los niveles, permitiendo que los paquetes de datos sean enrutados de origen a destino a través de la red.
	\item \textit{BGP (Border Gateway Protocol)}: Facilita la comunicación entre diferentes redes de nivel troncal, especialmente en entornos intercontinentales.
	\item \textit{OSPF (Open Shortest Path First)}: Protocolo de enrutamiento de estado de enlace utilizado en redes internas para determinar la mejor ruta para los datos.
\end{itemize}

\subsection*{3. En relación a la administración y operación de Internet, ¿qué rol tienen las siguientes organizaciones?}

\begin{itemize}
	\item \textbf{Internet Society (ISOC)}: Promueve el desarrollo abierto, la evolución y el uso de Internet como una infraestructura técnica global y un recurso para enriquecer la vida de las personas, actuando como una fuerza para el bien en la sociedad.
	\item \textbf{Internet Architecture Board (IAB)}: Supervisa el diseño técnico y la arquitectura de Internet, proporcionando dirección técnica a largo plazo para asegurar que Internet continúe creciendo y evolucionando como plataforma para la comunicación e innovación global.
	\item \textbf{Internet Engineering Steering Group (IESG)}: Responsable de la gestión técnica de las actividades del Internet Engineering Task Force (IETF) y del proceso de estándares de Internet.
	\item \textbf{Internet Engineering Task Force (IETF)}: Desarrolla y promueve estándares voluntarios de Internet, como los protocolos TCP/IP, mediante procesos abiertos para mejorar el funcionamiento de Internet.
	\item \textbf{Internet Assigned Number Authority (IANA)}: Administra la asignación de direcciones IP (IPv4, IPv6), números autónomos, el sistema de nombres de dominio (DNS), y la resolución inversa.
	\item \textbf{InterNIC}: Fue el servicio encargado de registrar nombres de dominio antes de que sus funciones fueran asumidas por la ICANN en 1998.
	\item \textbf{Latin American and Caribbean ccTLDs Organization (LACTLD)}: Agrupa a los administradores de dominios de nivel superior con código de país (ccTLD) en América Latina y el Caribe, representando su voz colectiva.
	\item \textbf{LACNIC (Latin American and Caribbean Internet Addresses Registry)}: Gestiona la asignación de direcciones IP y números autónomos para América Latina y el Caribe.
	\item \textbf{NIC Mexico}: Administra el dominio de nivel superior .mx, que es el código de dos letras asignado a México según el estándar ISO 3166.
\end{itemize}

\subsection*{4. Defina cada uno de los siguientes niveles de dominio y especifique cómo se administran}

\begin{itemize}
	\item \textbf{Generic Top Level Domains (gTLDs)}: Son dominios de nivel superior genéricos como .com, .net, .org, entre otros. Son gestionados por diversas organizaciones y registradores acreditados bajo la supervisión de la Internet Corporation for Assigned Names and Numbers (ICANN). \cite{FAQsICANN}
	
	\item \textbf{Sponsored Top Level Domains (sTLDs)}: Son dominios de nivel superior patrocinados destinados a comunidades específicas, como .edu (instituciones educativas), .gov (gobierno), y .mil (militar). Están administrados por entidades específicas que representan y gestionan los intereses de las comunidades a las que sirven, bajo la supervisión de ICANN.
	
	\item \textbf{Country Code Top Level Domains (ccTLDs)}: Son dominios de nivel superior específicos de países, designados por dos letras según el estándar ISO 3166, como .mx (México), .uk (Reino Unido), .us (Estados Unidos). Cada ccTLD es administrado por organizaciones nacionales correspondientes en cada país, con la asignación inicial y supervisión técnica a cargo de la Internet Assigned Numbers Authority (IANA), que es parte de ICANN. \cite{95EstructuraJerarquica}.
\end{itemize}

\printbibliography
\end{document}	

