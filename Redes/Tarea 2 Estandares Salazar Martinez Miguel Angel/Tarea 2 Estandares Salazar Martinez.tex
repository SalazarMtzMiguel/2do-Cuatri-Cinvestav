\documentclass[12pt]{report}
\usepackage[spanish,es-nosectiondot,es-lcroman]{babel}
\usepackage{siunitx}
\usepackage[utf8]{inputenc}
\usepackage{amsmath}
\usepackage{float}
\usepackage{listings}
\usepackage{xcolor}
\usepackage{amssymb}
\usepackage{graphicx}
\usepackage{hyperref}
\usepackage{geometry}
%Configuracion para el . en decimales
\sisetup{output-decimal-marker = {.}}
% Configuración para el código
\lstset{
	language=Python,
	basicstyle=\ttfamily\footnotesize,
	numbers=left,
	numberstyle=\tiny\color{gray},
	stepnumber=1,
	numbersep=10pt,
	backgroundcolor=\color{white},
	showspaces=false,
	showstringspaces=false,
	showtabs=false,
	frame=single,
	rulecolor=\color{black},
	tabsize=4,
	captionpos=b,
	breaklines=true,
	breakatwhitespace=false,
	linewidth=\linewidth,
	keepspaces=true,
	columns=flexible,
	keywordstyle=\bfseries\color{blue},
	commentstyle=\itshape\color{lightgray},
	stringstyle=\color{red},
	escapeinside={\%*}{*)},
}

% Configuración de los márgenes
\geometry{
	left=2cm,   % Margen izquierdo
	right=2cm,  % Margen derecho
	top=2cm,    % Margen superior
	bottom=2cm  % Margen inferior
}

% Title Page
\title{
	\begin{center}
		Tarea 2\\
		Procesos de Estandarización de Protocolos\\
		Redes de computadoras y protocolos de comunicación I
		
	\end{center}
}
\author{Salazar Martinez Miguel Angel}
\begin{document}
	\renewcommand{\arraystretch}{1.3}
	
	\maketitle

\section*{Procesos de Estandarización y Organización de Documentos}

\subsection*{1. The International Organization for Standardization (ISO)}

\textbf{Organización de documentos:} \\
Los estándares ISO siguen una estructura organizada en secciones, que incluyen:
\begin{itemize}
	\item Título y número de la norma.
	\item Introducción: Proporciona el contexto y objetivos del estándar.
	\item Alcance: Define qué cubre y qué no cubre el estándar.
	\item Referencias normativas: Lista de documentos a los que se hace referencia.
	\item Términos y definiciones: Explicación de términos utilizados en el estándar.
	\item Requisitos: Contenido principal que detalla los requisitos específicos.
	\item Anexos: Información complementaria opcional.
\end{itemize}

\textbf{Proceso de creación/actualización:} \\
El proceso de creación o actualización de un estándar ISO se lleva a cabo en varias etapas:
\begin{enumerate}
	\item \textit{Propuesta inicial:} Un miembro presenta una nueva norma o una revisión.
	\item \textit{Aprobación del proyecto:} El comité técnico correspondiente vota sobre la propuesta.
	\item \textit{Desarrollo:} Se redacta un borrador del estándar.
	\item \textit{Comentario público:} Se somete a revisión pública y se recogen comentarios.
	\item \textit{Aprobación final:} Se realiza una votación final para aprobar el estándar.
	\item \textit{Publicación:} El estándar es publicado oficialmente.
\end{enumerate}

\textbf{Fuente:} \url{https://www.iso.org/directives-and-policies.html}

\subsection*{2. International Telecommunication Union (ITU-T)}

\textbf{Organización de documentos:} \\
Las recomendaciones ITU-T están organizadas en:
\begin{itemize}
	\item Título y número de la recomendación.
	\item Resumen: Breve descripción del contenido.
	\item Introducción: Contexto y propósito.
	\item Cuerpo principal: Requisitos y directrices.
	\item Anexos: Información adicional o ejemplos.
\end{itemize}

\textbf{Proceso de creación/actualización:} \\
El proceso sigue estas etapas:
\begin{enumerate}
	\item \textit{Propuesta inicial:} Presentación por parte de un miembro.
	\item \textit{Estudio:} Evaluación inicial por un grupo de estudio.
	\item \textit{Redacción:} Desarrollo del borrador de la recomendación.
	\item \textit{Consulta y revisión:} Revisión por expertos y miembros.
	\item \textit{Aprobación:} Aprobación por consenso.
	\item \textit{Publicación:} Publicación de la recomendación.
\end{enumerate}

\textbf{Fuente:} \url{https://www.itu.int/en/ITU-T/publications/Pages/recs.aspx}

\subsection*{3. Internet Engineering Task Force (IETF)}

\textbf{Organización de documentos:} \\
Los documentos de la IETF, conocidos como RFCs, incluyen:
\begin{itemize}
	\item Título y número del RFC.
	\item Resumen: Breve descripción del RFC.
	\item Introducción: Contexto y propósito.
	\item Cuerpo principal: Contenido detallado.
	\item Conclusiones: Resumen de hallazgos o recomendaciones.
	\item Referencias: Fuentes y documentación citada.
\end{itemize}

\textbf{Proceso de creación/actualización:} \\
El proceso de creación de un RFC es:
\begin{enumerate}
	\item \textit{Idea inicial:} Propuesta por un individuo o grupo.
	\item \textit{Redacción del borrador:} Creación del borrador inicial.
	\item \textit{Discusión y revisión:} Revisión por la comunidad IETF.
	\item \textit{Última llamada:} Consulta final antes de la publicación.
	\item \textit{Publicación:} Publicación oficial como RFC.
\end{enumerate}

\textbf{Fuente:} \url{https://www.rfc-editor.org/pubprocess/}

\subsection*{4. European Telecommunications Standards Institute (ETSI)}

\textbf{Organización de documentos:} \\
Los documentos ETSI incluyen:
\begin{itemize}
	\item Título y número de la norma.
	\item Alcance: Definición de lo que cubre la norma.
	\item Referencias: Documentos y normas citados.
	\item Términos y definiciones: Clarificación de terminología.
	\item Requisitos técnicos: Contenido principal.
\end{itemize}

\textbf{Proceso de creación/actualización:} \\
El proceso incluye:
\begin{enumerate}
	\item \textit{Propuesta:} Presentación de una nueva norma.
	\item \textit{Desarrollo:} Redacción del borrador.
	\item \textit{Consulta:} Revisión pública.
	\item \textit{Aprobación:} Aprobación formal.
	\item \textit{Publicación:} Publicación oficial.
\end{enumerate}

\textbf{Fuente:} \url{https://www.etsi.org/standards/standards-making}

\subsection*{5. Institute of Electrical and Electronics Engineers (IEEE)}

\textbf{Organización de documentos:} \\
Los estándares IEEE están estructurados en:
\begin{itemize}
	\item Título y número del estándar.
	\item Introducción: Contexto y objetivos.
	\item Cuerpo principal: Requisitos y especificaciones.
	\item Anexos: Información complementaria.
\end{itemize}

\textbf{Proceso de creación/actualización:} \\
El proceso es:
\begin{enumerate}
	\item \textit{Propuesta inicial:} Presentación del proyecto.
	\item \textit{Aprobación inicial:} Revisión por el comité.
	\item \textit{Desarrollo del borrador:} Redacción y revisión.
	\item \textit{Revisión y votación:} Evaluación por los miembros.
	\item \textit{Aprobación final:} Aprobación por el comité de revisión.
	\item \textit{Publicación:} Publicación oficial.
\end{enumerate}

\textbf{Fuente:} \url{https://standards.ieee.org/develop/}

\subsection*{6. Society of Automotive Engineers (SAE)}

\textbf{Organización de documentos:} \\
Los estándares SAE incluyen:
\begin{itemize}
	\item Título y número del estándar.
	\item Propósito y alcance: Definición del propósito del estándar.
	\item Referencias: Documentación citada.
	\item Métodos de ensayo: Procedimientos específicos.
\end{itemize}

\textbf{Proceso de creación/actualización:} \\
El proceso sigue estos pasos:
\begin{enumerate}
	\item \textit{Propuesta:} Presentación de un nuevo estándar.
	\item \textit{Revisión:} Evaluación inicial por un comité.
	\item \textit{Redacción:} Desarrollo del borrador.
	\item \textit{Consulta:} Revisión pública.
	\item \textit{Aprobación:} Votación y consenso.
	\item \textit{Publicación:} Publicación oficial.
\end{enumerate}

\textbf{Fuente:} \url{https://www.sae.org/standards/development}

	
\end{document}

